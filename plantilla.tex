% ==============================================================================
\documentclass[10pt]{report}
% ==============================================================================

%%% ========================================================================
\usepackage{amsmath,amssymb,amsfonts,latexsym,graphicx}
\usepackage[spanish, mexico]{babel}
\usepackage[latin1]{inputenc}
%%% ========================================================================

%%% ========================================================================
\usepackage{geometry}
\setlength{\topmargin}{-5em} %
\setlength{\headsep}{15pt} %
\setlength{\footskip}{25pt} %
\setlength{\textheight}{9.75in} %
\setlength{\textwidth}{6.5in} %
\setlength{\hoffset}{-15pt}
\setlength{\parskip}{1ex}
%%% ========================================================================

% ========================================================================
\usepackage{fancyhdr}
\pagestyle{fancy}
\renewcommand\headrulewidth{.1pt}
\fancyhf{}
\fancyhead[L]{\footnotesize\sffamily \doctitle \hfill \topictitle \hfill \thepage}
\renewcommand\footrulewidth{.1pt}
\fancyfoot[c]{}
\fancyfoot[R]{\footnotesize\sffamily \theauthor \hfill \course \hfill \period}

\def\doctitle{Control 1}
\def\theauthor{E. Barrios}
\def\period{\today}
\def\topictitle{\textbf{Aproximaciones}}
\def\course{\textsc{Estad�stica Matem�tica}}
%%% ========================================================================

%%%%%%%%%%%%%%%%%%%%%%%%%%%%%%%%%%%%%%%%%%%%%%%%%%%%%%%%%%%%%%%%%%%%%%%%%%%%%%%%
\begin{document}
%%%%%%%%%%%%%%%%%%%%%%%%%%%%%%%%%%%%%%%%%%%%%%%%%%%%%%%%%%%%%%%%%%%%%%%%%%%%%%%%

%-------------------------------------------------------------------------------
\begin{itemize}

\item {Rafael Arredondo Villa}
\item {Fernando Rodriguez Guevara}
\end{itemize}\vspace{-1em}
\rule{\linewidth}{.1pt}
\renewcommand{\baselinestretch}{1.1}\normalsize
\setlength{\parskip}{1ex}
%-------------------------------------------------------------------------------

\begin{enumerate}
\item[I)] Calcule $n!$; la correspondiente aproximaci�n de Stirling
$\big(S(n)\big)=n!\approx\sqrt{2\pi n}n^ne^{-n}$; la diferencia entre ellas $\big(D(n)=n!-S(n)\big)$; y la diferencia relativa $\big(DR(n)=D(n)/n!\big)$ para $n=1,2,\dots,12$.
\end{enumerate}

\begin{table}[h!tb]
	\caption{\label{tab:approxError} Error de aproximaci�n $D(n)$ de la aproximaci�n de \emph{Stirling} $S(n)$ al factorial $n!$.}

	\medskip

\centering
\begin{tabular}{|c|c|c|c|c|}
\hline
   & $n!$      & $S(n)$       & $D(n)$       & $D(n)/n!$ \\ \hline
1  & 1         & 9.221370e-01 & 7.786300e-02 & 0.0778630 \\
2  & 2         & 1.919004e+00 & 8.099560e-02 & 0.0404978 \\
3  & 6         & 5.836210e+00 & 1.637904e-01 & 0.0272984 \\
4  & 24        & 2.350618e+01 & 4.938249e-01 & 0.0205760 \\
5  & 20        & 1.180192e+02 & 1.980832e+00 & 0.0165069 \\
6  & 720       & 7.100782e+02 & 9.921815e+00 & 0.0137803 \\
7  & 5040      & 4.980396e+03 & 5.960417e+01 & 0.0118262 \\
8  & 40320     & 3.990240e+04 & 4.176045e+02 & 0.0103573 \\
9  & 362880    & 3.595369e+05 & 3.343127e+03 & 0.0092128 \\
10 & 3628800   & 3.598696e+06 & 3.010438e+04 & 0.0082960 \\
11 & 39916800  & 3.961563e+07 & 3.011749e+05 & 0.0075451 \\
12 & 479001600 & 4.756875e+08 & 3.314114e+06 & 0.0069188 \\ \hline
\end{tabular}
\end{table}


\begin{enumerate}
\item[III)] Calcule los errores de aproximaci�n de la distribuci�n normal a la distribuci�n del promedio para
tama�os de muestra n = 30, 100, 500, para las distintas leyes de probabilidad indicadas en la tabla 2.
\end{enumerate}

\begin{table}[h!tb]
	\caption{\label{tab:approxError} Error de aproximaci�n de la distribuci�n normal a la del promedio de muestras tama�o n para distintas distribuciones.}

	\medskip

\centering
\begin{tabular}{|c|c|c|c|c|}
\hline
$distribuci�n$                                                  & $par�metros$                                                           & $n = 30$    & $n = 100$  & $n = 500$  \\ \hline
Binomial                                                        & p = 0.5                                                                & .06421342   & 0.09244345 & 0.2667152  \\ \hline
Bin(25,p)                                                       & p = 0.7                                                                & 0.05650117  & 0.02914352 & 0.04866444 \\
                                                                & p = 0.9                                                                & 0.07182747  & 0.06686821 & 0.03044123 \\ \hline
Poisson                                                         & \lambda = 1                                             & 0.1566382   & 0.07402676 & 0.06627173 \\
Po(\lambda)                                      & \lambda = 4                                             & 0.1009734   & 0.01775369 & 0.05841158 \\
                                                                & \lambda = 8                                             & 0.05624036  & 0.04897173 & 0.1104501  \\ \hline
Normal                                                          & \mu = 2, \sigma� = 4                     & 0.01879754  & 0.02039017 & 0.03169079 \\ \hline
Gamma                                                           & \alpha = 1, \beta = 3                    & 0.09236766  & 0.07579904 & 0.08748673 \\
Ga(\alpha,\beta)                  & \alpha = 3, \beta = 1                    & 0.04308491  & 0.04777173 & 0.1347708  \\
                                                                & \alpha = 5, \beta = 5                    & 0.02959665  & 0.01707562 & 0.06010626 \\ \hline
Beta                                                            & \theta_{1} = 1, \theta_{2} = 1     & 0.009798973 & 0.02123416 & 0.03062727 \\
Beta(\theta_{1},\theta_{2}) & \theta_{1} = 1/2, \theta_{2} = 2   & 0.07246766  & 0.05977173 & 0.07699944 \\
                                                                & \theta_{1} = 3, \theta_{2} = 1/3   & 0.0872818   & 0.05304345 & 0.03355759 \\
                                                                & \theta_{1} = 1/2, \theta_{2} = 1/2 & 0.0308      & 0.05709665 & 0.1380969  \\ \hline
\end{tabular}
\end{table}

%%%%%%%%%%%%%%%%%%%%%%%%%%%%%%%%%%%%%%%%%%%%%%%%%%%%%%%%%%%%%%%%%%%%%%%%%%%%%%%%
\end{document}
%%%%%%%%%%%%%%%%%%%%%%%%%%%%%%%%%%%%%%%%%%%%%%%%%%%%%%%%%%%%%%%%%%%%%%%%%%%%%%%%
