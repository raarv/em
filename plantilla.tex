% ==============================================================================
\documentclass[10pt]{report}
% ==============================================================================

%%% ========================================================================
\usepackage{amsmath,amssymb,amsfonts,latexsym,graphicx}
\usepackage[spanish, mexico]{babel}
\usepackage[latin1]{inputenc}
%%% ========================================================================

%%% ========================================================================
\usepackage{geometry}
\setlength{\topmargin}{-5em} %
\setlength{\headsep}{15pt} %
\setlength{\footskip}{25pt} %
\setlength{\textheight}{9.75in} %
\setlength{\textwidth}{6.5in} %
\setlength{\hoffset}{-15pt}
\setlength{\parskip}{1ex}
%%% ========================================================================

% ========================================================================
\usepackage{fancyhdr} 
\pagestyle{fancy} 
\renewcommand\headrulewidth{.1pt} 
\fancyhf{}
\fancyhead[L]{\footnotesize\sffamily \doctitle \hfill \topictitle \hfill \thepage} 
\renewcommand\footrulewidth{.1pt} 
\fancyfoot[c]{} 
\fancyfoot[R]{\footnotesize\sffamily \theauthor \hfill \course \hfill \period}

\def\doctitle{Control 1}
\def\theauthor{E. Barrios}
\def\period{\today}
\def\topictitle{\textbf{Aproximaciones}}
\def\course{\textsc{Estad�stica Matem�tica}}
%%% ========================================================================

%%%%%%%%%%%%%%%%%%%%%%%%%%%%%%%%%%%%%%%%%%%%%%%%%%%%%%%%%%%%%%%%%%%%%%%%%%%%%%%%
\begin{document}
%%%%%%%%%%%%%%%%%%%%%%%%%%%%%%%%%%%%%%%%%%%%%%%%%%%%%%%%%%%%%%%%%%%%%%%%%%%%%%%%

%-------------------------------------------------------------------------------
\begin{itemize}
\renewcommand{\baselinestretch}{1}\normalsize
\item {\bf Fecha m�xima de entrega:} Mi�rcoles 16 de febrero @ 16:00 h
\end{itemize}\vspace{-1em}
\rule{\linewidth}{.1pt}
\renewcommand{\baselinestretch}{1.1}\normalsize
\setlength{\parskip}{1ex} 
%-------------------------------------------------------------------------------

\textbf{Determin�stico}

\begin{enumerate}
\item[I)] Calcule $n!$; la correspondiente aproximaci�n de Stirling
$\big(S(n)\big)=n!\approx\sqrt{2\pi n}n^ne^{-n}$; la diferencia entre ellas $\big(D(n)=n!-S(n)\big)$; y la diferencia relativa $\big(DR(n)=D(n)/n!\big)$ para $n=1,2,\dots,12$.
\end{enumerate}

\begin{table}[h!tb]
	\caption{\label{tab:approxError} Error de aproximaci�n $D(n)$ de la aproximaci�n de \emph{Stirling} $S(n)$ al factorial $n!$.} 
	
	\medskip
	
\centering 
\begin{tabular}{|c|c|c|c|c|}
\hline
$n$ & $n!$ & $S(n)$ & $D(n)$ & $D(n)/n!$ \\
\hline\hline
 1 & & & & \\
2 & & & & \\
3 & & & & \\
4 & & & & \\
5 & & & & \\
6 & & & & \\
7 & & & & \\
8 & & & & \\
9 & & & & \\
10 & & & & \\
11 & & & & \\
12 & & & & \\
\hline
\end{tabular}
\end{table}


\textbf{Simulaci�n}
Sea $X$ una variable aleatoria (v.a.) definida sobre un espacio de probabilidad $(\Omega,\mathbb{S},\Pr)$. Para $w\in\Omega$, $X(w)=x\in\mathbb{R}$ se dice una \emph{realizaci�n} de la variable $X$. Si uno construye un \emph{histograma} para \emph{muchas} realizaciones $x$, el diagrama describe la distribuci�n de $X$ aproximando $f$, la funci�n de densidad de probabilidad (f.d.p.) de $X$. Mientras mayor sea el n�mero de realizaciones de $X$ mejor ser� la aproximaci�n.


Considere la v.a. $X$ con f.d.p. $f$. Simule $N$ realizaciones de la variable y construya el correspondiente histograma.  Por ejemplo, para el caso de $X\sim\text{Gamma}(\alpha=2,\beta=1)$ con $N=20,000$ realizaciones, el c�digo usando \textsf{R} ser�a el siguiente y da lugar a la Figura \ref{fig:histGamma}.
\begin{verbatim}
	set.seed(20220118)
	N <- 20000
	x <- rgamma(N,shape=2,scale=1)
	out <- hist(x, xlab="observaciones", ylab="frecuencia", main="Distribuci�n Gamma")
\end{verbatim}

\begin{figure}[h!tb]
	\centerline{\includegraphics[width=3.0in]{./histGamma}} %
	\caption{\label{fig:histGamma} Histograma de 20000 realizaciones de una
		distribuci�n gamma de par�metros $\alpha=2$ y $\beta=1$.}
\end{figure}


%%%%%%%%%%%%%%%%%%%%%%%%%%%%%%%%%%%%%%%%%%%%%%%%%%%%%%%%%%%%%%%%%%%%%%%%%%%%%%%%
\end{document}
%%%%%%%%%%%%%%%%%%%%%%%%%%%%%%%%%%%%%%%%%%%%%%%%%%%%%%%%%%%%%%%%%%%%%%%%%%%%%%%%
